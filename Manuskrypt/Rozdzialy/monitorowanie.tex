\chapter{Monitorowanie procesu gojenia ścięgna Achillesa}
\section{Ścięgno Achillesa}
Ścięgno Achillesa, nazywane również ścięgnem piętowym, jest największym i najsilniejszym ścięgnem występującym w ciele ludzkim. Stanowi wspólne zakończenie mięśnia trójgłowego łydki, w którego skład wchodzą dwie głowy mięśnia brzuchatego i mięsień płaszczkowaty. Całość struktury zlokalizowana jest w tylnym, powierzchownym przedziale łydki, co zostało przedstawione na Rysunku \ref{muscle_structure}.  
\begin{figure}[h!]
\centering
\includegraphics[width=0.55\textwidth]{figures/muscleStructure.jpg}
\caption{Lokalizacja mięśnia trójgłowego łydki wraz ze ścięgnem Achillesa.}
\label{muscle_structure}
\end{figure}
Z obu głów (brzuścców) mięśnia brzuchatego łydki wyrasta jedno szerokie, płaskie ścięgno, które jest początkiem części brzuchatej ścięgna Achillesa. Następnie ścięgno to łączy się z włóknami pochodzącymi od mięśnia płaszczkowatego, które układają się stycznie do wcześniej powstałej struktury. Wówczas kształt ulega stopniowemu zwężeniu i zaokrągleniu, aż do punktu o minimalnej szerokości (około 4 cm nad przyczepem dolnym [1]). W rejonie samego przyczepu dolnego znajdującego się na tylnej powierzchnia kości piętowej, ścięgno ponownie jest płaskie i szerokie.

W kolejnych podsekcjach szczegółowo omówiona została anatomia ścięgna Achillesa, jego biomechanika, potencjalne urazy wraz z czynnikami im sprzyjającymi oraz proces gojenia i możliwości jego wspomagania. Wszystkie te aspekty są istotne z uwagi na możliwości monitorowania procesów fizjologicznych występujących w ścięgnie. 
\subsection{Anatomia}
Średnia długość ścięgna Achillesa to 15 cm (11 - 26 cm). Średnia szerokość w rejonie początku wynosi 6.8 cm (4,5 - 8, 6 cm). Następnie, stopniowo ścięgno ulega zwężeniu do punktu o minimalnej szerokości 1.8 cm (1,2 - 2,6 cm). W rejonie samego przyczepu struktura ponownie się rozszerza i jej szerokość wynosi średnio 3.4 cm (2,0 - 4,8 cm) [2-3].
Zewnętrzną część ścięgna Achillesa stanowi ościęgno utworzone z tkanki łącznej włóknistej.
Achil
-Histologia
-Unaczynienie (krew, nerwy)

\subsection{Biomechanika}
\label{Biomechanika}
Zadaniem ścięgien jest transfer siły mięśniowej do układu szkieletowego.'

/*Hill - model mięśnia/nerwy/wapń - prezentacja openSim*/

Biomechanical simulations are driven by data from mathematical models chosen for the specific application. In most of the models within the softwares the skeletal system is defined as a set of rigid bodies representing particular body segment connected together in a hierarchical tree where the connections (joints) are represented by a mathematical function which models particular motion. The muscles are usually represented by the Hill model or some of it’s variations \cite{}

Parameters that can be applied to the skeletal model are called the kinematics parameters e.g.: number of DOF (Degrees Of Freedom), bones and joints geometry. Parameters that are utilized directly for solving the equation of the forces distribution in the system are called the dynamics parameters. Currently the most complex model of the whole human musculoskeletal system is the full body model from AnyBody Technology \cite{}

\subsection{Urazy i czynniki im sprzyjające}
\subsection{Leczenie, fazy gojenia i rehabilitacja}
\label{gojenie}
/* fizjoterapia */
/* fizykoterapia - fala, amplituda, częstotliwość*/ 

\section{Zastosowanie rezonansu magnetycznego}
Obrazowanie metodą \textit{Magnetycznego Rezonansu}, w skr. \textit{MR} (ang. \textit{Magnetic Resonance Imaging}, w skr. \textit{MRI}) jest metodą bazującą na odpowiednim odczycie reakcji atomów na silne pole magnetyczne. Zrozumienie tych procesów było efektem prac związanych z mechaniką kwantową takich jak: Einstein, Edwin Shrodinger, Heisenberg czy Pauli /*uzupełnij z licznych książek - krótką historię*/. Wymienione prace z początku XX wieku umożliwiły rozwój technologii MR zainicjowane zrozumieniem w latach 30-tych magnetycznej natury \textit{protonów}, cząstek subatomowych, które wraz z \textit{neutronami} wchodzą w skład jądra atomowego (zob. [1933 - stern gerlach]).

Dokładniej rzecz biorąc, główną własnością jądra atomowego umożliwiającą działanie MR jest \textit{spin}. Można powiedzieć, że jądro posiada spin jeżeli nie ma jednocześnie parzystej liczby protonów i parzystej liczby neutronów (por. [Rezonans magnetyczny]). Jądra o parzystej liczbie atomowej mają w \textit{stanie podstawowym} (charakteryzującym się najniższą energią) spin całkowity, a o nieparzystej połówkowy.

Cząstki o niezerowym spinie pochłaniają fale elektromagnetyczne o specyficznej częstotliwości konwertując energię na ruch rotacyjny nazywany \textit{rezonansem magnetycznym}. Zjawisko to opisuje się równaniem Larmora \ref{MRLarmor}:
\begin{equation}
\label{MRLarmor}
\omega_0 = \gamma \ast \beta_0,
\end{equation}
gdzie $\omega_0$ to prędkość obrotowa protonów (tzw. częstotliwość Larmora), $\gamma$ to współczynnik żyromagnetyczny właściwy danemu protonowi, a $\beta_0$ to wartość zadanego pola magnetycznego. Dla przykładu częstotliwość Larmora dla protonu wodoru, gdzie $\gamma$=42,57 MHz/T dla pola 1,5T równa jest 63,8 MHz. Wodór jest najczęściej wykorzystywanym pierwiastkiem w metodzie rezonansu magnetycznego, gdyż jest obecny w 99,98\% ludzkich tkanek (za [MR]). Jednak w szczególnych przypadkach stosuje się również obrazowania z wykorzystaniem częstotliwości odpowiednich dla fosforu, sodu czy węgla (zob. []).

Rezonujące protony mają własności podobne do magnesów prętowych i mogą być ukierunkowane przy pomocy pola magnetycznego. Protony o wysokich poziomach energii odchylają się przeciwnie do kierunku zadanego pola magnetycznego tzw. \textit{protony antyrównoległe}, a o niskich poziomach energii odchylają się zgodnie z kierunkiem pola tzw. \textit{protony równoległe}. W przeciętnym wycinku ciała ludzkiego jest więcej protonów równoległych niż antyrównoległych. Przykładowo w objętości 1 $\times$ 1 $\times$ 5 mm jest to liczba około 2 $\times$ $10^15$ cząstek. Różnica ta tworzy całkowitą magnetyzację $M_0$ danego obszaru.

Przy pomocy sterowanych impulsów o częstotliwości fal radiowych (tzw. $impulsy RF$) wykonywana jest zmiana $M_0$ obszaru, tworząca sygnał MR. Impulsy RF zawierają przestrzennie zmieniające się pole magnetyczne różnicujące częstotliwość i fazę precesji protonów, co wykorzystywane jest do kodowania lokalizacji przestrzennej $x$, $y$ sygnału. Kodowanie warstwy $z$ odbywa się na podstawie ustawiania częstotliwości Larmora dla danego obszaru.

Nieprzetworzona informacja z kierunków $x$, $y$, $z$ nazywana jest \textit{przestrzenią K} i koduje ona składniki częstotliwości sygnału. Do transformacji z przestrzeni częstotliwości na przestrzeń obrazu używana jest operacja matematyczna zwana \textit{odwrotną transformacją Fouriera}. Przykład takiego działania można znalexć w [http://mriquestions.com/fourier-transform-ft.html].

Z uwagi na różne parametry mierzone w reakcji protonów na zadane pole magnetyczne powstało wiele \textit{sekwencji MR} oraz ich składowych nazywanych w tej pracy \textit{protokołami MR}. Do badań opisanych w dalszej części tego manuskryptu zostało użytych 10 często stosowanych wariantów pracy rezonansu magnetycznego. Zostaną one pokrótce scharakteryzowane poniżej:
\begin{itemize}
	\item \textit{T1} -- mierzona jest \textit{relaksacja T1}, a zatem czas potrzebny protonom na powrót do stanu początkowego po odchyleniu o 90$^\circ$. Sygnał rezonansu magnetycznego $MR_s$ można obliczyć z zależności:
	\begin{equation}
		MR_s \sim \gamma_{pd} \ast [1-e^{-TR/T1}],
	\end{equation}
	gdzie $\gamma_{pd}$ to gęstość protonowa tkanki, $TR$ to \textit{czas repetycji} określany przez użytkownika.
	\item \textit{T2} -- mierzona jest \textit{relaksacja T2}, tj. czas potrzebny do utraty spoistości pomiędzy między spinami. Dokładniej mówiąc mierzone są różnice w częstotliwości Larmora powstające w czasie, wynikające z umiejscowienia protonów w różnych ośrodkach i niejednorodności w pobudzeniach polem magnetycznym. $MR_s$ przyjmuje wówczas następującą zależność:
	\begin{equation}
	\label{T2ecquation}
	MR_s \sim \gamma_{pd} \ast [1-e^{-TE/T2}],
	\end{equation}
	gdzie $TE$ to \textit{czas echa} zdefiniowany przez użytkownika.
	\item \textit{PD} (od ang. \textit{Proton Density}) -- mierzona jest gęstość spinowa tkanki wprost proporcjonalna do liczby protonów. Wówczas:
	\begin{equation}
	MR_s \sim \gamma_{pd}.
	\end{equation}
	\item \textit{T2 mapping} -- $T2$ z sygnału \ref{T2ecquation} mierzona jest dla wielu $TE$, co pozwala na interpolacje między wynikami z własnym doborem parametrów (zob. [Regulski]).
	\item \textit{T2 $^\ast$ GRE} (od ang. \textit{Gradient Echo}) -- w sekwencjach gradientowych protony odchylane są o kąt mniejszy od $90^\circ$ (zazwyczaj 10$^\circ$ -- 80$^\circ$). Odchylenia protonów występują w sekwencji prowadząc do cyklicznego przefazowania spinów. Dzięki temu uzyskuje się skupione echo gradientowe w czasie $TE$, w którym sygnał $MR_s$ jest zależny od obu czynników $T1$ i $T2$:
	\begin{equation}
	MR_s \sim \gamma_{pd} \ast [1-e^{-TE/T2}][1-e^{-TR/T1}],
	\end{equation}
	przy czym dla mniejszych kątów wkład $T2$ rośnie w stosunku do wkładu $T1$.
	\item \textit{T2 $^\ast$ GRE TE\_MIN} (od ang. \textit{Minimal Time Echo}) -- mierzone jest $T2$ z sygnału \ref{T2ecquation} przy minimalnym czasie $TE$ i tylko dla fragmentu przestrzeni K. 
	\item \textit{3D FSPGR} (od ang. \textit{Fast Spoiled Gradient Echo}) -- jest to szybka sekwencja gradientowa, wykorzystująca echo szczątkowe oraz \textit{spoiler}, tj. metodę do rozbicia wszelkiej pozostałej magnetyzacji na końcu każdego cyklu przefazowania. W tej pracy użyto czterech protokołów tej sekwencji. Wszystkie bazują na różnicy w częstotliwościach rezonansowych protonów wchodzących w skład tłuszczu i wody. Dokładniej:
	\begin{itemize}
		\item In Phase Ideal -- sygnał mierzony w czasie, gdy protony należące do tłuszczu i wody są w zgodnej fazie;
		\item Out Phase Ideal -- sygnał mierzony w czasie, gdy protony należące do tłuszczu i wody są w antyfazie;
		\item Fat Ideal -- sygnał mierzony przy maksymalnej fazie protonów tłuszczu i minimalnej protonów wody;
		\item Water Ideal -- sygnał mierzony przy maksymalnej fazie protonów wody i minimalnej protonów tłuszczu.
	\end{itemize}
\end{itemize}

Zdecydująca część protonów wodoru znajduje się w tkankach miękkich człowieka dlatego rezonans magnetyczny może być wykorzystany do monitorowania gojenia się ścięgien i więzadeł. W szczególności obserwowana jest ciągłość ścięgna w płaszczyznie sagitalnej, uszkodzenia śródścięgniste objawiające się przerwaniem w naturalnym warkoczu ułożonym z tkanek, pogrubienia ścięgna w szczególności oceniane w przekrojach axialnych jednorodność wrzecionowatość i inne nieregularności kształtu, ostrość ścięgna i jego rozgraniczenie od tkanek otaczających i zmiany na brzegach, obrzęk w okalających ścięgno tkankach, jednorodność charakteryzująca się podobieństem przekrojów sąsiednich, liczbę ewentualnych zrostów.

\section{Zastosowanie ultrasonografii}

Kolejną z metod obrazowania medycznego jest \textit{Ultrasonografia}, w skr. \textit{USG} (ang. \textit{Ultrasonography}, \textit{US}). Bazuje ona na efektach związanych z propagacją w tkankach \textit{ultradźwięków}, tj. fal akustycznych o częstotliwościach powyżej 20 kHz.

Propagacja fal w przyrodzie była tematem rozważań myślicieli takich jak Pitagoras, Arystoteles czy Galileusz, którzy ugruntowali pole badań pod kolejne osiągnięcia matematyczno-inżynieryjne. W tej kwestii, do jednego z przełomów doszło w 1822 roku, kiedy to szwajcarski inżynier Daniel Colladen oraz matematyk Charles-Francois Sturm wyznaczyli przybliżoną prędkość rozchodzenia się fali akustycznej w wodzie. Badanie wykonano na Jeziorze Genewskim symultanicznie mierząc czas jaki potrzebny był dźwiękowi podwodnego wystrzału i sygnałowi dzwonka rozchodzącego się w powietrzu do przebycia drogi pomiędzy dwoma łódkami oddalonymi o 10 mil. Wyliczona wartość wyniosła wówczas 1435 m/s nie różniąc się znacząco od dzisiaj przyjmowanej estymacji równej 1480 m/s. Analogicznie wyliczane są obecnie prędkości rozchodzenia się w fali w innych ośrodkach takich jak powietrze, tkanki miękkie, kości itp.

58 lat później, w 1880 roku bracia Curie opisali w \cite{Curie1880} \textit{efekt piezoelektryczny}, czyli zjawisko polegające na pojawieniu się ładunku elektrycznego pod wpływem naprężeń mechanicznych w krysztale o anizotropowej budowie, takiej jak ma np. kwarc. W przypadku odwrotnym, przyłożenie napięcia do odpowiedniego kryształu generuje drgania. 

Efekty te są wykorzystywane w \textit{głowicy aparatu usg}, przyrządu do generowania i odbierania ultradźwięków. Przykładowo, polaryzowanie kryształu piezoelektrycznego krótkim impulsem elektrycznym $\sigma_1(t)$ pobudza go do drgań na własnej częstotliwości rezonansowej. Zakładając, że kryształ ma kształt walca o grubości $d$=0,64mm, to będzie stanowił rezonator półfalowy, w którym wystąpi drganie rezonansowe o długości fali $\lambda = 2d$ czyli $\lambda = 1,28$ mm. Jeżeli wykonany jest z tytanianu baru, dla którego prędkość propagacji drgań wynosi $c$=4460m/s, to częstotliwość drgań własnych tego
kryształu wyniesie:
\begin{equation}
f = \frac{c}{\lambda} = \frac{4460 m/s}{1,28 mm} = 3,5 MHz.
\end{equation}
Odwrotnie, powracająca fala tzw. \textit{echo} wygeneruje impuls elektryczny $\sigma_2(t)$ w skutek drgań wywołanych w krysztale. 
 
Echo jest naturalnie falą różniącą się od sygnału nadawanego, a zmiany te są w przeważającym stopniu efektem zjawisk znanych z optyki falowej tj. \textit{odbicia}, \textit{załamania}, \textit{dyfrakcja}, \textit{rozpraszanie} i \textit{pochłanianie} (zob. []).

Zależą one on częstotliwości fali, która propaguje się w ośrodku o pewnej \textit{impedancji akustycznej ośrodka} $Z$ wyrażanej jako:
\begin{equation}
Z = \rho c = \sqrt{\epsilon \rho},
\end{equation}
gdzie $c$, to prędkość rozchodzenia się fali, $\rho$ to gęstość ośrodka, a $\epsilon$ to \textit{moduł odkształcalności objętościowej}, tj. parametr opisujący jak zmieni się objętość ośrodka pod danym ciśnieniem. Parametry wybranych ośrodków zestawiono w Tabeli \ref{USG-params} 
\begin{figure}[h!]
	\centering
	\includegraphics[width=0.55\textwidth]{figures/USG-params.png}
	\caption{Parametry ośrodków często mierzonych w badaniach USG.}
	\label{USG-params}
\end{figure}

Dla przykładu, do zjawiska załamania lub odbicia dochodzi kiedy fala pada na granice dwóch ośrodków o różnych impedancjach akustycznych $Z_1$ i $Z_2$. Zależność ta opisana jest prawem Snella:
\begin{equation}
R = \frac{I_r}{I_0} = \left(\frac{Z_1-Z_2}{Z_1+Z_2}\right)^2,
\end{equation}
gdzie $I_r$ to natężenie fali padającej, a $I_0$ odbitej. Natomiast $R$, czyli współczynnik odbicia, jest parametrem, który rośnie wraz ze wzrostem kąta odchylenia od kierunku prostopadłego, aż do całkowitego odbicia.

Z kolei do rozpraszania bądz pochłaniania fali dochodzi kiedy to fala pokonuje daną drogę w ośrodku o pewnej $Z$, co zapisywane jest następująco:
\begin{equation}
I=I_0 \epsilon^{-\gamma x},
\end{equation}
gdzie $\gamma$ to współczynnik osłabienia zależny od $Z$, a $x$ to droga przebyta przez falę. Efekt ten można korygować poprzez dobór odpowiedniego $I_0$.

Analiza amplitudy i częstotliwości sygnału nadanego i echa umożliwia rekonstrukcję obrazu USG. W przypadku najczęściej stosowanych w praktyce rekonstrukcji przekrojów dwuwymiarowych (tzw. tryb B) współczesny tor budowania prezentacji wizualnej (tzw. \textit{beamforming}) wygląda następująco: 
\begin{enumerate}
	\item Głowica ultradźwiękowa emituje impulsy w postaci wąskiej wiązki w ściśle określonym kierunku. Wiązka zawiera sygnał z $N$ przetworników zawierających kryształy piezoelektryczne.
	\item Echa z danego kierunku pozwalają na obliczenie pojedynczego promienia akustycznego, który jest iloczynem charakterystyk nadawanego i odbieranego sygnału.
	\item Wszystkie promienie, których we współczesnych aparatach może być do kilkuset (zob. [GE Voluson]), służą do formowania obrazu, który tworzony jest we \textit{współrzędnych biegunowych} ($r$, $\theta$) w przypadku głowic mechanicznych sektorowych, wieloelementowych convex czy fazowych lub we współrzędnych prostokątnych ($x$, $y$) w przypadku głowic mechanicznych lub wieloelementowych liniowych\footnote{szczegółowy opis głowic USG i ich charakterystyk można znalezć w []}. 
\end{enumerate}

Tryb B umożliwia również wizualizację obrazów dynamicznych. Przykładowo jeżeli na obraz składa się 400 promieni i każdy odsłuchiwany jest do głębokości 15cm, to czas gromadzenia danych dla ośrodka o średnim c=1500 m/s wynosi $2\frac{2\times15}{1500 m/s}\times400 = 0,08 s$, czyli 12 obrazów na sekundę. Częstotliwość tę można zwiększać, zmniejszając liczbę promieni lub głębokość obserwacji.

Innym często stosowanym trybem rekonstrukcji obrazu (wykorzystywanym również w tej pracy) jest tryb D bazujący na \textit{efekcie Dopplera}, do którego dochodzi w przypadku przechodzenia fali przez ośrodki względem siebie się przesuwające. Zmienia się wówczas częstotliwość fali, co wyrażone jest następującym wzorem:
\begin{equation}
f_r = 2 f_o\frac{v}{c}\cos(\theta),
\end{equation} 
gdzie $f_r$ to zmiana częstotliwości fali nadawanej $f_0$, zależna od kąta $\theta$ pomiędzy falą i ośrodkiem poruszającym się i prędkościami rozchodzenia się fali w obu ośrodkach tj. $v$ i $c$. Tryb D jest wykorzystywany w praktyce np. do monitorowania przepływu krwi w tkankach. 

W kontekście ścięgna Achillesa tryb B jest użyteczny do obrazowania tkanek miękkich. Przydatna jest zwłaszcza możliwość zobrazowania ukierunkowania struktur włókien ścięgnistych na podstawie czego radiolog może wnioskować o fazie gojenia. Składowa czasowa jest interesująca z perspektywy fizjoterapeuty oceniającego m.in. ślizg w ścięgnie przy wykonywaniu odpowiednich ruchów np. zginania podeszwowo-grzbietowego stopy. Tryb D natomiast może służyć do oceny unaczynienia ścięgna w kolejnych etapach gojenia, które jak wiadomo z sekcji \ref{gojenie} zmienia się w czasie.

W porównaniu do rezonansu magnetycznego USG charakteryzuje się niskimi kosztami aparatury. Uzyskiwane obrazy są jednak trudniejsze w interpretacji, co przekłada się na koszty wyszkolenia kadry. W tym kontekście przydatne mogą okazać się nowe rozwiązania w warstwie sprzętowej i oprogramowania. 

Do pierwszej grupy należy zaliczyć zastąpienie przetworników z piezoelektrykami, przetwornikami budowanymi w technologi MEMS np. cMUT, czy pMUT (zob. [butterfly-http://news.mit.edu/2018/startup-butterfly-network-ultrasound-smartphone-0207]) oraz układy pozwalające przetwarzać surowy sygnał ultradxwiękowy (zob. [us4us]). Do drugiej grupy należą algorytmy sztucznej inteligencji pozwalające wydobyć i zinterpretować interesującą informację z niskiej jakości obrazów (np. zob. [nvidia-keynot-clarisa-project]). 

Badania obrazowe nie są jedynymi, które służą do oceny gojenia się ścięgna Achillesa. W kolejnej sekcji zostały opisane metody oceny biomechaniki, które samodzielnie jak i w połączeniu z analizą obrazową stanowią wartościową informację diagnostyczną.

\section{Zastosowanie badań biomechanicznych}

W poprzednich sekcjach zostały opisane dwie najczęstsze metody monitorowania procesu gojenia się ścięgna Achillesa z udziałem badań obrazowych. W tym celu stosuje się ocenę \textit{biomechaniki}, która bada właściwości mechaniczne elementów składowych organizmów żywych takich jak tkanki czy narządy. W szczególności mierzone parametry pozwalają na \textit{ocenę funkcjonalną}, a zatem w jakim stopniu dany element może realizować swoją funkcję.

Funkcja ścięgna Achillesa została opisana w \ref{Biomechanika}. Do jej oceny stosuje się szereg badań funkcjonalnych do których należą również pomiary wykorzystane w tej pracy opisane w punktach poniżej:

\begin{enumerate}
	\item ATRS Achilles Tendon Total Rupture Score (ATRS) -- oceniającej poziom ograniczenia, z którymi borykają się w następstwie urazu w skali od 0 do 10 (gdzie 0 oznacza znaczne ograniczenie, a 10 brak ograniczeń/objawów) [Nilsson-Helander i wsp. 2007, Bąkowski i wsp. 2017]. Skala ATRS: skala od 1 do 100 pkt.
	\item Pomiar stabilograficzny na platformie dynamometrycznej -- platforma HUR. Badania odbywały się boso z oczami otwartymi. Wykonane zostały dwie próby po 30 sekund kolejno na prawej i lewej kończynie dolnej. Następnie pacjent przechodził na kolejną platformę do stabilografii dynamicznej (Biodex Balance System), gdzie miał za zadanie utrzymanie równowagi na niestabilnym podłożu. Wykonano 3 próby po 30 sekund na prawej i potem lewej kończynie dolnej (ustawienie platformy na poziomie 2). Wyniki zostają porównane między kończynami. Stabilografia statyczna: pomiar drogi wychylenia środka masy pacjenta w trakcie stania jednonóż na platformie dynamometrycznej [mm].
	\item analiza na ścieżce podometrycznej -- Badano rozkład sił nacisków podeszwowej strony stóp na podłoże. Zebrano pomiar podczas stania swobodnego, wspięć na palce oraz przysiadu boso bez odrywania pięt. Następnie dokonano analizy chodu (3 przejścia) i biegu (5 przebiegnięć) boso oraz w obuwiu sportowym. rotacja podudzia [deg], długość kroków [cm], faza podparcia [\%], faza przenoszenia [\%], max heel force [N] oraz max toe force [N]
	\item Test skoczności i mocy (tzw. siły dynamicznej) kończyn dolnych - wyskoki pionowe z miejsca, wykonywane na platformie dynamometrycznej ("JBA" Zb. Staniak, Polska). Wykonano po dwie próby obunóż oraz na prawej i lewej kończynie dolnej w obuwiu sportowym. W celu pełnego zaangażowania kończyn dolnych zalecono pacjentowi podczas badania aby trzymał ręce na biodrach. Zmierzono moc maksymalną $Pmax$ i średnią $Pm$, maksymalną wysokość uniesienia $hmax$ i obniżenia $k$ środka masy ciała przed odbiciem.
	\item Pomiary maksymalnych momentów sił mięśni zginaczy podeszwowych i grzbietowych stawu skokowego. Ze względu na dwustawową funkcję ścięgna Achillesa badanie przeprowadzono w dwóch pozycjach: z wyprostowanym (ryc.1) oraz zgiętym do 50 stopni stawem kolanowym (ryc.2) [Orishimo i wsp. 2008] w warunkach izometrii i izokinetyki w trzech prędkościach kątowych 60$^\circ$/s (5 powtórzeń), 120$^\circ$/s (8 powtórzeń) oraz 180$^\circ$/s (10 powtórzeń) przy wykorzystaniu urządzenia Humac Norm (USA). Przed badaniem osoba odbywała 5 minutową rozgrzewkę na steperze. Do analizy brano wartości maksymalne pomiarów. Wyniki zostały porównane pomiędzy operowaną a zdrową kończyną dolną. Maksymalne wartości momentu siły mięśni zginaczy podeszwowych i grzbietowych stawu skokowego w warunkach izometrii i izokinetyki w dwóch pozycjach: z wyprostowanym oraz zgiętym do 50 stopni stawem kolanowym [Nm] oraz deficyt pomiędzy operowaną i zdrową kończyną dolną [\%]
\end{enumerate}

Powyższy protokół badań został wdrożony w placówce Carolina Medical Center (grupa Luxmed) [CMC] i został przez ortopedów i fizjoterapeutów określony jako wystarczający do oceny przywracania funkcji ścięgna Achillesa po rekonstrukcji. 

Istnieją jednak możliwości rozszerzenia tak zdefiniowanego podejścia, aż do wersji maksimum, która uwzględniałaby pełne modelowanie układu mięśniowo-szkieletowego. Dane wówczas muszą pochodzić z możliwie dokładnych urządzeń pomiarowych takich jak:
\begin{itemize}
	\item \textit{Komputerowa analiza ruchu} (ang. \textit{Motion Capture}) -- narzędzie wykorzystujące systemy czujników do zapisu informacji o zmianach położenia obiektu rejestrowanego np. pacjenta. Do wiodących rozwiązań należy zaliczyć systemy firmy Vicon [] lub BTS [].
	\item \textit{Płyty dynamometryczne} (ang. \textit{Force Plates}) -- narzędzie wykorzystywane do pomiaru sił reakcji podłoża w trzech prostopadłych płaszczyznach. Dzięki temu można określić sumaryczny udział mięśni w generowaniu sił odpowiadających za balans ciała, ruch w danym kierunku oraz przeciwstawianie się sile grawitacji. Do wiodących rozwiązań należą płyty firmy Kistler [].
	\item \textit{Elektromiografia}, w skr. EMG (ang. \textit{Electromyography}) -- narzędzie do pomiaru pobudzeń poszczególnych grup mięśniowych podczas ruchu. Wykorzystywane jest m.in. do określenia rozkładu sił zmierzonych przez płyty dynamometryczne na poszczególne mięśnie.
	\item Badania uzupełniające -- do tej kategorii należy uwzględnić informacje pozwalające na uszczegółowienie modelu mięśniowo-szkieletowego np.: wymiary poszczególnych segmentów ciała (np. goleń, udo, tors) tzw. pomiary antropometryczne; Maksymalne siły izometryczne mierzone w systemach takich jak Biodex; geometrie kości mierzone np. z pomocą Tomografii Komputerowej [] ew. MR; lokalizacje przyczepów mięśniowych określaną przy pomocy MR lub USG; środek masy poszczególnych segmentów określany np. przy użyciu badania DXA (od ang. Double X Ray Absorption); kompozycja włókien mięśniowych widoczna w USG.
\end{itemize}

Przy czym parametry komputerowej analizy ruchu, sił reakcji podłoża i elktromiografii zbierane są synchronicznie podczas ruchu, natomiast badania uzupełniające nie. 

Z uwagi na dużą liczbę możliwych do zmierzenia parametrów, ich integracja odbywa się w modelach komputerowych zaimplementowanych w różnego rodzaju oprogramowaniu do symulacji biomechanicznych. Do najczęściej używanych modeli należą Gait22 [], Gait 20 [] oraz obecnie najbardziej złożony AnyBody full body model []. Historia komputerowo wspomaganego, kompleksowego modelowania biomechaniki ruchu zaczęła się we wczesnych latach 1990, kiedy to Delp i Loan przedstawili oprogramowanie SIMM [SIMM]. Obecnie SIMM jak również inne oprogramowania komercyjne takie jak: Visual 3D (Cmotion Inc.) [], Anybody (Anybody Technology) [], czy Adams (MSC Software Corp.) [], dostarczają narzędzi do wartościowych symulacji np. chodu [], biegu [] jak również konsekwencji różnych zabiegów chirurgicznych [] i chorób []. Istnieją również narzędzia otwarte, do których należą szeroko wykorzystywany OpenSim [] rozwijany na Uniwersytecie w Stanford, czy też Human Motion [] wywodzący się z instytutu badawczego RIKEN w Japonii.


Most accurate dynamic data of human motion kinematics and dynamics are gathered in a specially equipped Motion Laboratories. Such facilities consists of synchronized instruments for motion capture (e.g. Vicon or BTS systems), external forces measurements (e.g. force plates) and electromyography (EMG) for muscles activity monitoring. Additionally the model can be parameterized with further parameters like: body segments dimensions and weight collected with use of the anthropometry techniques; maximal isometric muscle forces measured with systems like Biodex; geometry of bones derived from the computer tomography; locations of muscle attachments founded in the Magnetic Resonance Imaging (MRI) scans; center of mass of the particular body segment assessed based on the Double X Ray Absorption (DXA) studies; the composition of the particular muscles viewed in ultrasonography etc.






 

\section{Inne metody}