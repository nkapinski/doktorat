%%%%%%%%%%%%%%%%%%%%%%%%%%%%%%%%%%%%%%%%%%%%%%%%%%%%%%%%%%%%%%%%%%%%%%%%%%%
% This is a sample header for a sample dissertation. Fill in the name,
% and the other information. LaTeX will work out the table of
% content, the list of figures and of tables for you.
%%%%%%%%%%%%%%%%%%%%%%%%%%%%%%%%%%%%%%%%%%%%%%%%%%%%%%%%%%%%%%%%%%%%%%%%%%%

\newpage
\thispagestyle{empty}


% ******* Title page *******
% **************************

\begin{onehalfspacing}
\begin{center}

\centering
\includegraphics[keepaspectratio,scale=0.1]{./figures/Logo-IBIB.jpg} \\[.5cm]


{\fontsize{17}{17}\selectfont
\textsc{Polska Akademia Nauk \\[.3cm]
Instytut Biocybernetyki i Inżynierii Biomedycznej  \\[1.7cm]
%Kierunek Biotechnologia  \\[2.5cm]
}

\textbf{Praca doktorska \\[1.7cm]}}



\large 
{Proces gojenia ścięgna Achillesa oceniany przez fuzję danych z wykorzystaniem głębokich sieci neuronowych} \\[2.3cm]
% Jeśli tytuł pracy zajmuje 2 linijki, wartość [2.3cm] zamieniamy na [3.1cm], jeśli tylko jedną - na [3.9cm] i odwrotnie - zwiększając liczbę linijek o jedną (do czterech) zmieniamy na [1.5cm] itd.


\large
\begin{flushleft}
Autor: mgr inż. Norbert Kapiński  \\
Kierujący pracą:  dr hab. inż. Antoni Grzanka \\
Promotor pomocniczy:  dr Jakub Zieliński \\
\end{flushleft}

\vspace{1.5cm}
Warszawa, wrzesień 2019
\end{center}
\end{onehalfspacing}

\singlespacing
\newpage
\thispagestyle{empty}
\mbox{}


%ABSTRACT
\begin{abstract}
\textbf{Proces gojenia ścięgna Achillesa oceniany przez fuzję danych z wykorzystaniem głębokich sieci neuronowych.}
\newline\newline
W pracy została zaproponowana automatyczna metoda oceny stopnia gojenia się ścięgna Achillesa widocznego w badaniach rezonansu magnetycznego. Do opracowania metody  posłużyły wybrane algorytmy sztucznej inteligencji oraz przetwarzania obrazów.
Z ich wykorzystaniem został zaproponowany i zaimplementowany algorytm, który na podstawie danych wejściowych w postaci pojedynczego badania generuje ankietę składającą się z sześciu parametrów ocenionych w skali 0--7. Badania przedstawione w tej pracy pozwoliły również na optymalny wybór sekwencji rezonansu magnetycznego, co pozwoliło na skrócenie czasu badania z początkowej godziny do pięciu minut.  Wyniki tej pracy zostały zastosowane do opisu parametrycznego ścięgna Achillesa w projekcie START w kooperacji Interdyscyplinarnego Centrum Modelowania Matematycznego i Komputerowego Uniwersytetu Warszawskiego oraz Carolina Medical Center. Rezultaty są również przygotowywane do wdrożenia w ramach grantu z programu Inkubator Innowacyjności 2.0.
\newline
\begin{center}
	\textbf{Summary}
\end{center}
\textbf{Assessment of Achilles tendon healing process with the use of data fusion and deep neural networks.}
\newline\newline
This work introduces a novel approach for the automatic assessment of the Achilles tendon healing process visible in Magnetic Resonance Imaging. The presented method benefits from recent achievements in artificial intelligence and computer vision. For a single Magnetic Resonance study, the algorithm generates a quantitative description consisted of 6 parameters assessed on a scale from 0 to 7. Furthermore, in the result of the presented studies, the input of the method is acquired in approximately five minutes, which represents a significant speedup comparing to the initial hour. Results of the work were used in the START project led by the Interdisciplinary Centre for Mathematical and Computational Modelling of the University of Warsaw and Carolina Medical Center as well as will benefit further from the Inkubator Innowacyjności 2.0 program in order to be prepared for commercialization.

\end{abstract}
%END OF ABSTRACT


\doublespacing
\newpage
\thispagestyle{empty}
\mbox{}

%\pagestyle{empty}
\pagenumbering{Roman}
\setcounter{page}{0} \pagestyle{plain}


\tableofcontents

\listoffigures
\listoftables



\pagestyle{fancy}