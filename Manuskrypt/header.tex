%%%%%%%%%%%%%%%%%%%%%%%%%%%%%%%%%%%%%%%%%%%%%%%%%%%%%%%%%%%%%%%%%%%%%%%%%%%
% This is a sample header for a sample dissertation. Fill in the name,
% and the other information. LaTeX will work out the table of
% content, the list of figures and of tables for you.
%%%%%%%%%%%%%%%%%%%%%%%%%%%%%%%%%%%%%%%%%%%%%%%%%%%%%%%%%%%%%%%%%%%%%%%%%%%

\newpage
\thispagestyle{empty}


% ******* Title page *******
% **************************

\begin{onehalfspacing}
\begin{center}

\centering
\includegraphics[keepaspectratio,scale=0.1]{./figures/Logo-IBIB.jpg} \\[.5cm]


{\fontsize{17}{17}\selectfont
\textsc{Polska Akademia Nauk \\[.3cm]
Instytut Biocybernetyki i Inżynierii Biomedycznej  \\[1.7cm]
%Kierunek Biotechnologia  \\[2.5cm]
}

\textbf{Praca doktorska \\[1.7cm]}}



\large 
{Proces gojenia ścięgna Achillesa oceniany przez fuzję danych z wykorzystaniem głębokich sieci neuronowych} \\[2.3cm]
% Jeśli tytuł pracy zajmuje 2 linijki, wartość [2.3cm] zamieniamy na [3.1cm], jeśli tylko jedną - na [3.9cm] i odwrotnie - zwiększając liczbę linijek o jedną (do czterech) zmieniamy na [1.5cm] itd.


\large
\begin{flushleft}
Autor: mgr inż. Norbert Kapiński  \\
Kierujący pracą:  dr hab. inż. Antoni Grzanka \\
Promotor pomocniczy:  dr Jakub Zieliński \\
\end{flushleft}

\vspace{1cm}
Warszawa, wrzesień 2019
\end{center}
\end{onehalfspacing}

\singlespacing
\newpage
\thispagestyle{empty}
\mbox{}


%ABSTRACT
\begin{abstract}
\textbf{Proces gojenia ścięgna Achillesa oceniany przez fuzję danych z wykorzystaniem głębokich sieci neuronowych.}
\newline\newline
W pracy została zaproponowana automatyczna metoda oceny stopnia gojenia się ścięgna Achillesa na podstawie wyników badania rezonansem magnetycznym. Do opracowania metody posłużyły wybrane algorytmy sztucznej inteligencji oraz przetwarzania obrazów.
Z ich wykorzystaniem wykonana została ekstrakcja klasycznych cech obrazowych oraz cech wyliczonych na podstawie modeli głębokich sieci neuronowych. Fuzja powyższych podejść umożliwiła automatyczne wygenerowanie ankiet składająca się z sześciu parametrów ocenianych w skali 0--7, opisujących pojedyncze badania pacjentów. W porównaniu z oceną eksperta radiologa w 5-ciu na 6 parametrów uzyskano wyniki oceny charakteryzujące się absolutnym błędem średnim poniżej 1 i korelacją w zakresie 0,65--0,85. Badania przedstawione w tej pracy pozwoliły również na optymalny wybór sekwencji rezonansu magnetycznego i 12-sto krotne skrócenie czasu akwizycji. Finalne rezultaty zostały zastosowane do opisu parametrycznego ścięgna Achillesa w projekcie START, realizowanego w kooperacji Uniwersytetu Warszawskiego oraz Carolina Medical Center, jak również przygotowywane są do wdrożenia w ramach grantu z programu Inkubator Innowacyjności 2.0.
\newline
\begin{center}
	\textbf{Summary}
\end{center}
\textbf{Assessment of Achilles tendon healing process with the use of data fusion and deep neural networks.}
\newline\newline
This work introduces a novel approach for an automatic assessment of the Achilles tendon healing process based on Magnetic Resonance Imaging. The presented method benefits from recent achievements in artificial intelligence and computer vision. Particularly the algorithm combains clasical image processing features with trainable ones and for a single Magnetic Resonance study generates a quantitative description consisted of 6 parameters assessed on a scale from 0 to 7. Comparing to an expert radiologists assessment the automatic approach, for 5 of 6 parameters, results in a mean absolute error less than 1 and correlation in range of 0.65--0.85. Furthermore, research presented within the work allowed for optimal selection of magnetic resonance sequences and a 12-fold reduction in acquisition time. Final results were used in the START project led by the University of Warsaw and Carolina Medical Center as well as will benefit further from the Inkubator Innowacyjności 2.0 program in a commercialization process.

\end{abstract}
%END OF ABSTRACT


\doublespacing
\newpage
\thispagestyle{empty}
\mbox{}

%\pagestyle{empty}
\pagenumbering{Roman}
\setcounter{page}{0} \pagestyle{plain}


\tableofcontents

\listoffigures
\listoftables



\pagestyle{fancy}