\chapter{Uzasadnienie wyboru tematu pracy}
Wraz z występującym w ostatnich latach systematycznym wzrostem liczby obrazowań medycznych uwidacznia się potrzeba na komputerowe wspomaganie pracy radiologów oceniających badania obrazowe. W szczególności zastosowanie znajdują aplikacje usprawniające generowanie raportów, rozwiązania do personalizacji diagnostyki i narzędzia poprawiające jej jakość.

Niniejsza praca, w odpowiedzi na powyższe zagadnienia, przedstawia propozycję strukturyzacji i automatyzacji oceny gojenia ścięgna Achillesa widocznego w obrazowaniu Rezonansem Magnetycznym (w skr. RM). Badanie to, w kontekście przedmiotowego ścięgna, jest dokładną metodą wykorzystywaną do oceny zmian strukturalnych i morfologicznych w zakresie tkanek miękkich. Wedle obecnych standardów ocena tego badania jest subiektywna i niesparametryzowana, a zatem stanowi ciekawy temat badawczy związany z możliwościami komputerowego wspomagania radiologów i usprawnienia ich pracy.

-- DL
-- dostęp do unikatowego zbioru danych
-- podsumowanie wyboru
 

\chapter{Cele i hipoteza pracy}
\chapter{Struktura pracy}
\chapter{Zbiór danych i metody badawcze}
\chapter{Charakterystyka i wyniki przeprowadzonych badań}
\chapter{Wnioski końcowe}