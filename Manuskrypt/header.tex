%%%%%%%%%%%%%%%%%%%%%%%%%%%%%%%%%%%%%%%%%%%%%%%%%%%%%%%%%%%%%%%%%%%%%%%%%%%
% This is a sample header for a sample dissertation. Fill in the name,
% and the other information. LaTeX will work out the table of
% content, the list of figures and of tables for you.
%%%%%%%%%%%%%%%%%%%%%%%%%%%%%%%%%%%%%%%%%%%%%%%%%%%%%%%%%%%%%%%%%%%%%%%%%%%

\newpage
\thispagestyle{empty}


% ******* Title page *******
% **************************

\begin{onehalfspacing}
\begin{center}

\centering
\includegraphics[keepaspectratio,scale=0.1]{./figures/Logo-IBIB.jpg} \\[.5cm]


{\fontsize{17}{17}\selectfont
\textsc{Polska Akademia Nauk \\[.3cm]
Instytut Biocybernetyki i Inżynierii Biomedycznej  \\[1.7cm]
%Kierunek Biotechnologia  \\[2.5cm]
}

\textbf{Praca doktorska \\[1.7cm]}}



\large 
{Proces gojenia ścięgna Achillesa oceniany przez fuzję danych z wykorzystaniem głębokich sieci neuronowych} \\[2.3cm]
% Jeśli tytuł pracy zajmuje 2 linijki, wartość [2.3cm] zamieniamy na [3.1cm], jeśli tylko jedną - na [3.9cm] i odwrotnie - zwiększając liczbę linijek o jedną (do czterech) zmieniamy na [1.5cm] itd.


\large
\begin{flushleft}
Autor: mgr inż. Norbert Kapiński  \\
Kierujący pracą:  dr hab. inż. Antoni Grzanka \\
Promotor pomocniczy:  dr Jakub Zieliński \\
\end{flushleft}

\vspace{1.5cm}
Warszawa, wrzesień 2019
\end{center}
\end{onehalfspacing}

\singlespacing
\newpage
\thispagestyle{empty}
\mbox{}


%ABSTRACT
\begin{abstract}
The abstract will go here.... \\
W tym miejscu można umieścić abstrakt pracy. W przeciwnym wypadku należy usunąć/zakomentować ninijeszy fragment kodu.
\end{abstract}
%END OF ABSTRACT


\doublespacing
\newpage
\thispagestyle{empty}
\mbox{}

%\pagestyle{empty}
\pagenumbering{Roman}
\setcounter{page}{0} \pagestyle{plain}


\tableofcontents

\listoffigures
\listoftables



\pagestyle{fancy}