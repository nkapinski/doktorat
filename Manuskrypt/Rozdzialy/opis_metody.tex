\chapter{Nowa metoda oceny procesu gojenia ścięgna Achillesa}
\label{NewMethod}

%ATRS \cite{Kearney2012}, VISA-A \cite{Robinson2001} or FAOS \cite{Roos2001} - dorzuć do biomechaniki.

%	Medical Imaging - especially Magnetic Resonance Imaging (MRI) and Ultrasonography (US) - is one of the most common technique that nowadays is used to monitor soft tissues. Some papers e.g.  \cite{Khan2003, Ibrahim2013} indicate the advantage of MRI over US showing that MRI appearance can be easier associated with the clinical outcomes. Despite this, US also present in e.g. \cite{vanSchie2009, PradoCosta2018} as a valuable technique for the soft tissues properties investigation, yet technical issues and lack of standardization limits its use.} - dorzuć do USG/MRI

W tym rozdziale zostanie zaprezentowana autorska propozycja metody oceny procesu gojenia się ścięgna Achillesa bazująca na badaniach obrazowych. W szczególności przedstawiony zostanie sposób na ilościowy opis tego procesu widocznego w obrazach Rezonansu Magnetycznego i Ultrasonografii oraz metoda automatycznego wyliczania wskazanych w opisie parametrów. 

W chwili pisania tej pracy nie istnieje wedle wiedzy autora wystandaryzowany sposób oceny 

%\textcolor{blue}{The perturbations in the healing process of the Achilles tendon may result from the surgery, further rehabilitation and other factors like diet, obedience to treatment guidelines and patient predispositions. Thus, challenges in the assessment process are connected to the proper recognition of the morphological changes of the tendon structure, its functional limitations, and individual features.}

%\textcolor{blue}{


%\textcolor{blue}{Particularly there is no structured description that could apply to MRI and US studies of the Achilles tendon. The existing approaches like ATRS  \cite{Kearney2012}, VISA-A \cite{Robinson2001} or FAOS \cite{Roos2001} are only suitable for measuring the general outcome, related to symptoms and physical activity of patients. The motivation for our work is to utilize medical imaging techniques to improve precise monitoring and comparison of different therapies for ruptured Achilles tendon.}

%\textcolor{blue}{Thus we performed a broad study on 60 patients who underwent the open surgical reconstruction. The patients were monitored over a year of rehabilitation with the use of 10 MRI and 2 US protocols, acquired in 10 properly distributed time-steps. We also gathered a homogeneous control group composed of 29 healthy volunteers. To our best knowledge, there is no such study comparable in scale to the one that we introduce in this paper. For example, in\cite{Tam2017} the conclusions are formulated based on two samples, in \cite{Fujikawa2007} the authors evaluate a total of 30 healing tendons repaired with a percutaneous surgical technique and 10 repaired with an open surgical one, finally in \cite{Albano2017} the authors present results for 43 patients to study the significance of growth factors.}

istnieją niszowe metody np. Majometr [], ale nie ma jeszcze nakładki obiektywizującej rezonans i USG
\section{Metodyka}
\section{Rozróżnienie ścięgna zdrowego i po zerwaniu}
\section{Obliczanie krzywych gojenia}
\subsection{Topologia sieci}
\subsection{Redukcja wymiarowości}
\subsection{Miara wygojenia}