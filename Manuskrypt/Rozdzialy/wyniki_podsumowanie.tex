\chapter{Wyniki i walidacja}


\section{Ocena procesu gojenia z użyciem nowej metody}

-- porównanie z end-to-end
-- porównanie z samym DL

%			\hline
%\multirow{3}{*}{baseline} 
%& MAE & $1.26\pm{0.44}$ & $0.71\pm{0.12}$ & $0.7\pm{0.16}$ & $0.97\pm{0.26}$ & $0.92\pm{0.29}$ & $0.99\pm{0.25}$ \\
%&MAX-AE & 3.53 & 2.49 & 1.91 & 2.34 & 2.2 & 2.47\\
%&Corr & 0.58 & 0.47 &-0.07 & 0.60 & 0.56 & 0.58\\
%\hline
%\multirow{3}{*}{\begin{tabular}{@{}c@{}} baseline \\ limited \end{tabular}} 
%& MAE & $1.24\pm{0.16}$ & $0.82\pm{0.09}$ & $0.75\pm{0.08}$ & $1.06\pm{0.10}$ & $0.90\pm{0.09}$ & $0.96\pm{0.10}$ \\
%&MAX-AE & 3.54 & 2.46 & 1.82 & 2.70 & 2.13 & 2.18\\
%&Corr   & 0.61 & 0.64 &-0.08 & 0.55 & 0.55 & 0.65\\
%\hline


-- wykresy
-- macierz błędu (confusion matrix)


\section{Porównanie z wynikami ultrasonografii}
\section{Porównanie z wynikami badań biomechanicznych}


\chapter{Podsumowanie}
%- jeden radiolog więc ankieta ok
%- AlexNet uczy się bardziej generycznych cech niż ResNet, który zwraca uwagę na szczególy: https://www.researchgate.net/post/Can_AlexNet_be_a_better_feature_extractor_than_ResNet

%https://icmlviz.github.io/icmlviz2016/assets/papers/4.pdf